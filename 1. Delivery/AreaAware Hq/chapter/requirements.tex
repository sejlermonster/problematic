\label{chp_requirements}
\chapter{Requirements}
The AreaAware HQ system creates situational awareness for Emergency Commanders. This is secured by giving the Emergency Commander a demografic overview of the situation. The overview show registered events from Emergency Responders and other Emergency Commanders. The system also facilities fast and secure communication with Emergency Responders.

The AreaAware system realizes that a civil crisis is of great importance and that securing fast and safe communication throughout the crisis is of the highest priority. The AreaAware system secures the best network coverage throughout the world by supporting both SINE/TELTRA Network and commercial internet. A crisis situation can result in the commercial internet not being operational. By using both SINE and commercial internet AreaAware makes sure that communication between Emergency Commanders and Emergency Responders can always get through. The SINE network also allows for prioritizing communication and will secure communication between Emergency Commanders and Emergency Responders.

\section{Scope}
The goal of the system is to handle civilian crisis's in a more effective way. This should be done by giving the Emergency Commander a Common Operational Picture to give situational awareness and support the Emergency Commander in the better communication with Emergency Responders in the field

\subsection{Identification}
To support the goal defined in the scope AreaAware Hq will be developed. Based on the goal a system that gives the Emergency Commander a common operational picture (COP) will be described throughout this document. The COP should give the Emergency Commander situational awareness and the ability to command Emergency Responders during a civilian crisis. 

This section includes a full identification of the system-of-interest to which this document applies.
\myFigure{actorContextDiagram.png}{Actor context diagram}{asdfq}{0.7} 

AreaAware Hq will only include a development of the Common operation picture which will communicate with devices already used in civil crisis scenarios. AreaAware Hq will communicate over the defined standard network used in these scenarios. This network is called SINE(Sikkerhednetværket) and it follows an international standard for communication called TETRA which is used during crisis situations world wide. 

\subsection{System overview}
The system will facilitate the Emergency Commander in:
\begin{itemize}
	\item Give a demographic overview of the crisis area
	\item Let the Emergency Commander locate events, things of interest and actors in the area.
	\item Let the Emergency Commander send commands to actors in the area.
\end{itemize}

The system developed will contain of:
\begin{itemize}
	\item A web application that can be accessed over the internet
	\item Communication Modem that connects already used computer system by Emergency Commanders to both SINE network and Internet.
\end{itemize}

AreaAware HQ will be accessed as a web application from systems already used by the Emergency Commander at the mobile headquarter. This will allow the Emergency Commander to gain situational awareness through a structured map that show relevant events for the Emergency Commander. The System will also allow for communication to Emergency Responders who are working in the field during a civil crisis.

As a part of the AreaAware HQ a Communication Modem is developed. This allows already used computer systems at the mobile headquarters to gain access to both the SINE network and internet access. The Communication model will prioritize communication on the available network with the best bandwidth. This allows for a network coverage during a civil crisis through the SINE network but also allow for higher networks speeds through the internet if possible.

\subsection{Document overview}
\label{sec_documentOverview}
This document will describe the functional and non-functional requirement of the system of interest. 

The functional requirements are described using Task descriptions in section \ref{sec_functional}. These requirements describes what the system should be able to do. Task descriptions includes a Task name, the purpose of the Task and might include a precondition for the Task. The purpose should reflect the goals stated. The Tasks are furthermore described with sub-tasks that elaborates on the behaviour of the system and user.  Variants are described to explain alternatives to the main sub-tasks. \citep{taskDescription}

The non-functional requirements are divided up in sub-sections and categorized. The following sections describes the non-functional requirements of the system: \ref{sec_nonFunc1}, \ref{sec_nonFunc2}, \ref{sec_nonFunc3}, \ref{sec_nonFunc4}, \ref{sec_nonFunc5}, \ref{sec_nonFunc6} and \ref{sec_nonFunc7} 
 

%\section{Required states and modes}
%If the system is required to operate in multiple states or modes, these states and modes shall be defined.

\newpage
\label{sec_functional}
\section{System capability requirement}
This section specifies requirements on the behaviour of the system. The behaviour is described with Task descriptions explained in document overview \ref{sec_documentOverview}.

\subsection{TA-01}
\begin{longtable}{| p{2.5cm}  | p{10cm} |  }
	\hline
	Task & Demographic overview of crisis situation  \\
	Purpose &  Give Emergency Commander situational awareness through a demographic overview of the crisis situation \\
	Trigger/ &  \\ Precondition &  \\
	\hline
	Sub-Tasks: & \\
	1. & Access demografic overview of crisis situation \\
	\hline
	2. & Use the demografic overview to gain situational awareness of the crisis situation \\
	\hline
	3. & Use the demografic overview to better command the crisis situation \\
	\hline
\end{longtable}


\subsection{TA-02}
\begin{longtable}{| p{2.5cm}  | p{10cm} |  }
	\hline
	Task & Distribute information over text \\
	Purpose & Send information over text to the platforms employed by the Emergency Responders \\
	Trigger/ &  \\ Precondition &  \\
	\hline
	Sub-Tasks: & \\
	1. & Write text message\\
	2. & Choose criteria \\
	3. & Distributes message to various actors \\
	\hline
	Variants: & \\
	2a. & Geography criteria chosen. This selects Emergency Responders in a specific area \\
	2b. & Role criteria chosen. This selects Emergency Responders with a specific role\\
	2c. & Identity criteria chosen. This select an Emergency Responder with a specific identity \\
	\hline
\end{longtable}

\newpage
\subsection{TA-03}
\begin{longtable}{| p{2.5cm}  | p{10cm} |  }
	\hline
	Task & Distribute information over voice \\
	Purpose & Send information over voice to the platforms employed by the Emergency Responders \\
	Trigger/ &  \\ Precondition &  \\
	\hline
	Sub-Tasks: & \\
	1. & Choose filtering criteria for filtering Emergency Responders \\
	3. & Distributes information over voice to the Emergency Responders \\
	\hline
	Variants: & \\
	2a. & Geography criteria chosen. This selects Emergency Responders in a specific area \\
	2b. & Role criteria chosen. This selects Emergency Responders with a specific role\\
	2c. & Identity criteria chosen. This select an Emergency Responder with a specific identity \\
	\hline
\end{longtable}

\subsection{TA-04}
\begin{longtable}{| p{2.5cm}  | p{10cm} |  }
	\hline
	Task & Distribute information over voice and video \\
	Purpose & Send information over voice and video to the platforms employed by the Emergency Responders \\
	Trigger/ &  \\ Precondition &  \\
	\hline
	Sub-Tasks: & \\
	1. & Choose filtering criteria for filtering Emergency Responders \\
	3. & Distributes information over voice and video to the Emergency Responders \\
	\hline
	Variants: & \\
	2a. & Geography criteria chosen. This selects Emergency Responders in a specific area \\
	2b. & Role criteria chosen. This selects Emergency Responders with a specific role\\
	2c. & Identity criteria chosen. This select an Emergency Responder with a specific identity \\
	\hline
\end{longtable}

\newpage
\subsection{TA-05}
\begin{longtable}{| p{2.5cm}  | p{10cm} |  }
	\hline
	Task &  Present coordinate of each Emergency Responder on COP \\
	Purpose & Give the Emergency Commander situational awareness over the resources of the Emergency Responders in the crisis area. \\
	Trigger/ &  \\Precondition & An Emergency Responder is active in the crisis area\\
	\hline
	Sub-Tasks: & \\
	1. & The coordinate of the Emergency Responder is show on the demografic overview \\
	\hline
	2. & Choose to focus on the Emergency Responders \\
	\hline
	3. & Name, Role and ID of the Emergency Responder becomes available \\
	\hline
\end{longtable}

*Custom event is an event type created by the Emergency Commander. This lets the Emergency Commander create custom event types that fits each civilian crisis. The Emergency Responders can then choose to register a created event type. Each custom event is related to an event name and a GPS coordinate related to the event.


\subsection{TA-06}
\begin{longtable}{| p{2.5cm}  | p{10cm} |  }
	\hline
	Task & Create Custom event type and register an event as a Emergency Responder \\
	Purpose & Let the Emergency Commander create custom event types to be registered by Emergency Responders. This lets the Emergency Commander choose relevant event types for each civilian crisis.  \\
	Trigger/ &  \\Precondition & An  Emergency Responder registers event using currently employed platform \\
	\hline
	Sub-Tasks: & \\
	1. & Create Custom event type \\
	2. & Register an observation of the created custom event type \\
	3. & The observation of the custom event shows up on the demographic overview at the coordinate of registration. \\
	\hline
\end{longtable}

\newpage
\subsection{TA-07}
\begin{longtable}{| p{2.5cm}  | p{10cm} |  }
	\hline
	Task & The Emergency Commander must be able to focus on various sources and types of information  \\
	Purpose & Let the Emergency Commander focus on different types of information to better command the situation \\
	Trigger/ &  \\ Precondition & Events or information must registered \\
	\hline
	Sub-Tasks: & \\
	1. & Choose a custom event type* as focus option \\
	2. & The focus option is highlighted on the demographic overview \\
	\hline
\end{longtable}

*Custom event is an event type created by the Emergency Commander. This lets the Emergency Commander create custom event types that fits each civilian crisis. The Emergency Responders can then choose to register a created event type. Each custom event is related to an event name and a GPS coordinate related to the event.

\subsection{TA-08}

\begin{longtable}{| p{2.5cm}  | p{10cm} |  }
	\hline
	Task & Review the history of events \\
	Purpose & Present the event history for the Emergency Commander \\
	Trigger/ &  \\ Precondition & An Emergency Responders has already registered an event \\
	\hline
	Sub-Tasks: & \\
	1. & Click on a registered event \\
	2. & Present list of events including Names, gps coordinate and even type  \\
	\hline
\end{longtable}

\FloatBarrier
\newpage

\label{sec_nonFunc1}
\section{System external interface requirements}
This section will describe the system’s external interfaces. Interfaces will be assigned a project-unique identifier. 

The system will communicate over the network used by the Emergency Responders during a civil crisis. This network is called SINE. The interface will be provided by SINE and AreaAware HQ will follow the given interface to allow for SINE certification. AreaAware HQ will also make it possible to extend the capabilities of the SINE network by using the internet to extend the network. The AreaAware Hq Communication Model will be used to extend the possibilities of SINE and also use the internet to communicate with Emergency Responders.

The diagram shown on figure \ref{fig:externalInterface} explains the external interfaces of AreaAware Hq.

\myFigure{InterfaceDiagram.png}{External interfaces}{fig:externalInterface}{1} 

\begin{longtable}{| p{3.2cm} |  p{10cm} | }
	\hline
	\textbf{Requirement id} &  \textbf{Requirement } \\
	\hline
	NF-01 & The system will follow the interface defined by the SINE network.  \\
	\hline
	NF-02 & The system will extend the capabilities of the SINE network by also using the internet for communication  \\
	\hline
\end{longtable}

%\section{System internal interface requirements}
%This paragraph shall be divided into subparagraphs to specify the requirements, if any, for the system’s internal interfaces. It shall state if there are internal interfaces left to the design or to system requirement specifications for components. Furthermore, the description in item 3.3 also applies to internal interfaces

%\subsection{Subcontracter to optional thingy...}

%\section{System internal data requirements}
%Requirements on data internal to the system such as databases, data files etc. shall be stated. If there are internal interfaces left to the design or to system requirement specifications for components these shall be identified.

%\section{Adaption requirements}
%Requirements on installation-dependent data and operational parameters that the system requires.

%\section{Safety requirements}
%System requirements concerning minimizing unintended hazards to personnel, property and the physical environment

\label{sec_nonFunc2}
\section{Secuirity and privacy requirements}
This section specifies the system requirement concerned with the maintaining security and privacy.

\begin{longtable}{| p{3.2cm} |  p{10cm} | }
	\hline
	\textbf{Requirement id} &  \textbf{Requirement } \\
	\hline
	NF-03 & All external communication from AreaAware Hq is encrypted.  \\
	\hline
\end{longtable}

\label{sec_nonFunc3}
\section{System environment requirements}
This paragraph shall specify the system requirements, if any, regarding the environment in which the system must operate.

\begin{longtable}{| p{3.2cm} |  p{10cm} | }
	\hline
	\textbf{Requirement id} &  \textbf{Requirement } \\
	\hline
	NF-04 & The COP system will be operational in temperatures from -10 degree celcius to 50 degree celsius.  \\
	\hline
\end{longtable}

\label{sec_nonFunc4}
\section{Computer resource requirements}
This section specifies resource utilization requirements. It will be further divided into hardware and communications etc.

Resource minimum requirements for external systems using the AreaAware HQ web application
\begin{longtable}{| p{3.2cm} |  p{10cm} | }
	\hline
	\textbf{Requirement id} &  \textbf{Requirement } \\
	NF-05 & CPU i3, 2.0 ghz \\
	\hline
	NF-06 & Intel HD 4000 \\
	\hline
	NF-07 & RAM 2 gb \\
	\hline
	NF-08 & 20 gb free hard disk space \\
	\hline
	NF-09 & Browser supporting ECMAScript 5 and HTML 5 must be installed on the computer. \\
	\hline
	NF-10 & Microphone \\
	\hline
	NF-11 & Web cam \\
	\hline
\end{longtable}


Relevant resource information about the AreaAware Communication Modem
\begin{longtable}{| p{3.2cm} |  p{10cm} | }
	\hline
	\textbf{Requirement id} &  \textbf{Requirement } \\
	\hline
	NF-12 & LAN with 16-p switch. Allows for connection up to connect 14 computers \\
	\hline
	NF-13 & Powered by 230 volt. \\
	\hline
\end{longtable}


Communication requirements for AreaAware HQ :
\begin{longtable}{| p{3.2cm} |  p{10cm} | }
	\hline
	\textbf{Requirement id} &  \textbf{Requirement } \\
	\hline
	NF-14 & AreaAware HQ is SINE certified.  \\
	\hline
	NF-15 & AreaAware HQ Communicates through the SINE network. \\
	\hline
	NF-16 & AreaAware HQ Communicates uses LTE to extend the capabilities of the SINE network \\
	\hline
	NF-17 & AreaAware HQ must have a connection available from either the SINE network or the internet with a download and upload bandwith of at least 100 kbit/s \\
	\hline
\end{longtable}


%\section{System quality factors}
%This paragraph specifies quantitative requirements concerning system functionality, reliability, maintainability, availability, flexibility, portability (software), reusability, testability, usability an other attributes.


\label{sec_nonFunc5}
\section{Design and construction constraints}
This section specifies the requirements, that constrain the construction of the system

AreaAware HQ Communication Modem Weight and dimension requirements:
\begin{longtable}{| p{3.2cm} |  p{10cm} | }
	\hline
	\textbf{Requirement id} &  \textbf{Requirement } \\
	\hline
	NF-18 & AreaAware HQ Communication Modem will weigh up to 3 kilograms inclusive box and cables.  \\
	\hline
	NF-19 & AreaAware HQ Communication Modem will have the following dimensions: 10cmx40cmx20cm. \\
	\hline
\end{longtable}

\label{sec_nonFunc6}
\section{Personnel-related requirements}
This paragraphs specifies the requirements, included to accommodate the number, skill levels, duty cycles, training needs, or other information about the personnel who will use or support the system.


\begin{longtable}{| p{3.2cm} |  p{10cm} | }
	\hline
	\textbf{Requirement id} &  \textbf{Requirement } \\
	\hline
	NF-20 &The customer will have 24x7 access to a Service Centre as stated in \emph{Software Warranty Service}.  \\
	\hline
	NF-21 & A user manual will be available for the Emergency Commander who uses the AreaAware Hq. \\
	\hline
	NF-22 & 2-day training will be available for every Emergency Commander who uses the AreaAware Hq. \\
	\hline
\end{longtable}



\subsection{Software Warranty Service consists of:}
Defect Reporting. For Critical Defects, the Customer will have 24x7 access to the Service Centre by e-mail or phone to request defect repair, as described below. “Critical Defect means that the
application is down or is at high risk, business functions cannot be conducted, or the Customer is experiencing continual failures or data corruption as a result of the defect. To report non-critical
defects, the Customer will have e-mail or phone access to the Service Centre during the Principal Period of Maintenance (”PPM”), which is 8:00 a.m. to 5:00 p.m., local time, Monday through Friday,
excluding local holidays. Defect Repair. Defect repair includes verification of the existence of a defect, determination of the
severity or impact of the defect, and determination of the conditions under which the defect may recur. 
The Company will, at its option: 
\begin{itemize}
\item For a Critical Defect, commence action within a 2-shift hour response window using commercially reasonable efforts to provide an immediate fix or temporary solution of, or workaround
to, the defect.
\item For a non-critical defect, commence action within an 8-shift hour response window to provide either the action described for a Critical Defect or a statement that the defect will be corrected
in a software product revision or a future software release.
\item Provide a statement that the Software operates as described in then-current user documentation or that the defect arises when such Software is used other than in a manner for which it
was designed. For Software added to an installed System, warranty service must be upgraded to the same software support plan, if any, as that of the Software already installed on that
System. Customer will pay the difference between standard warranty and upgraded warranty
service.
\end{itemize}


%\section{Training-related requirements}
%System requirements pertaining to training.

%\begin{itemize}
%	\item
%\end{itemize}


\section{Logistics-related requirements}
This paragraphs specifies the requirements, if  any, concerned with logistics considerations.

\begin{longtable}{| p{3.2cm} |  p{10cm} | }
	\hline
	\textbf{Requirement id} &  \textbf{Requirement } \\
	\hline
	NF-24 & Every event registration, start and end of information distribution and coordinates of Emergency Responders and must be logged in a database. All logged information will be related to specific personnel identifier and with a time stamp  \\
	\hline
\end{longtable}

\newpage
\label{sec_nonFunc7}
\section{Packaging requirements}
This paragraphs specifies the requirements, if any, for packaging, labling and handling the systems and its components for delivery.

\begin{longtable}{| p{3.2cm} |  p{10cm} | }
	\hline
	\textbf{Requirement id} &  \textbf{Requirement } \\
	\hline
	NF-23 & The COP is available as a web application in a browser supporting ECMAScript 5 and HTML5. \\
	\hline
\end{longtable}