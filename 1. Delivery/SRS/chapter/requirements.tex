\label{chp_requirements}
\chapter{Requirements}
This chapter on system requirements is organised into a number of sections. Both functional and non-functional requirements will be explained throughout this chapter.

Every requirement has a project-unique identifier to support testing and traceability.

\section{Required states and modes}
If the system is required to operate in multiple states or modes, these states and modes shall be defined.

\section{System capability requirement}
This paragraph specifies requirements on the behavior of the system and shall include applicable relevant parameters. It shall be divided further into subparagraphs depending on the system capability in question

\section{System external interface requirements}
This paragraphs shall be divided into subparagraphs to specify the requirements, if any, for the system’s external interfaces. Interfaces shall be assigned a project-unique identifier. One or more interface diagrams shall be provided.

\section{System internal interface requirements}
This paragraph shall be divided into subparagraphs to specify the requirements, if any, for the system’s internal interfaces. It shall state if there are internal interfaces left to the design or to system requirement specifications for components. Furthermore, the description in item 3.3 also applies to internal interfaces

\section{System internal data requirements}
Requirements on data internal to the system such as databases, data files etc. shall be stated. If there are internal interfaces left to the design or to system requirement specifications for components these shall be identified.

\section{Adaption requirements}
Requirements on installation-dependent data and operational parameters that the system requires.

\section{Safety requirements}
System requirements concerning minimizing unintended hazards to personnel, property and the physical environment

\section{Secuirity and privacy requirements}
This paragraph shall specify the system requirements, if any, concerned with the maintaining security and privacy.

\section{System enviroment requirements}
This paragraph shall specify the system requirements, if any, regarding the environment in which the system must operate.

\section{Computer resource requirements}
This paragraph specifies resource utilization requirements. It can be further divided into hardware, software, communications etc.

\section{System quality factors}
This paragraph specifies quantitative requirements con cerning system functionality, reliability, maintainability, availability, flexibility, porta bility (software), reusability, testability, usability an other attributes.

\section{Design and construction constraints}
This paragraph specifies the requirements, if any, that constrain the design and construction of the system

\section{Personnel-related requirements}
This paragraphs specifies the requirements, if any, included to accommodate the number, skill levels, duty cycles, training needs, or other information about the personnel who will use or support the system.

\section{Training-related requirements}
System requirements pertaining to training.

\section{Logistics-related requirements}
This paragraphs specifies the requirements, if  any, concerned with logistics considerations.

\section{Packaging requirements}
This paragraphs specifies the requirements, if any, for packaging, labling and handling the systems and its components for delivery.

\section{Other requirements}
Examples include requirements for system documentation
not covered in other contractural documents