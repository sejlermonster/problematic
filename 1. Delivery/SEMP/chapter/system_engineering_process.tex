%!TEX root ../semp.tex
%
\thispagestyle{fancy}
\chapter{System Engineering Process}
\label{chp:se_process}

%% SECTION Systems engineering process
\section{Systems engineering planning}

\subsection{Deliverables}
\label{plan:deliverables}
The milestone plan here will denote the deadline the delivery of system engineer documents for the AreaAware project, to its proper stakeholders.
A review will ensure all partners are satisfied with content of these documents.
In case of change of documents content, a new delivery and review should should take place.
\myTable{Plan for delivery of system engineer documents. The colored cells denote {\color{deadlinecolor} Deadline} and \color{reviewcolor} Review.}{tab:milestones}{table/milestone}

\subsection{Process inputs}
The input to the planning process is the project proposal \emph{Statement Of Work (SOW)}, which can be summed up to: \\
In a military battlefield the \emph{SitaWare Suite} provides the capability to establish situational awareness, what in other settings may be referred to as a common operations picture (COP).
The project AreaAware propose to extend these possibilities offered by \emph{SitaWare Suite} to cover civilian scenarios, where actors for police, armed forces, hospitals and emergency management work together to control a crisis situation.
A successful result of this could be a scenario where the commanders from a mobile headquarter can evaluate incoming information, act upon it and possibly dispatch orders or information to relevant actors to respond to by.

\subsection{Responsibilities and authorities}
The Project Manager for AreaAware has the authority and responsibility to appoint an AreaAware Chief Systems Engineer, who shall have the authority and responsibility to carry out the policies and activities described herein.\\\\
The AreaAware Chief Systems Engineer has the authority and responsibility to appoint Subsystem Engineers, who have primary responsibility for all subsystem level issues.
Thus the Subsystem Engineers have the authority and responsibility to ensure that the defined interfaces' requirements, for the specific subsystem, are met.\\\\
The AreaAware Chief Systems Engineer works with the Subsystem Engineers and subcontractors Systems Engineer to implement the systems engineering process as described herein across the collaboration to ensure that requirements and interfaces are fully defined and that all technical issues are efficiently resolved.\\\\
The approval of the AreaAware Chief Systems Engineer is required on all documents and decisions that have the potential to impact the ability of the delivered systems.

\subsection{Procedure and verification}
The AreaAware system will follow the 6 life cycle stages described in table \ref{tab:lifeCycle}.

\myTable{Life cycle stages, their purpose and decision gates.}{tab:lifeCycle}{table/lifeCycle}

The Vee model in figure \ref{fig:veeModel} is used visualize the system engineering focus of a projects' life cycle.
From the Vee model it should be clear that the definition of verification is in focus.
Such that the engineering process iterates to ensure that a concept or design is feasible, and that the stakeholders remain supportive of the solution in each stage of the projects life cycle.

\myFigure{figure/vee}{Vee Model of a projects' life cycle.}{fig:veeModel}{1}

\subsection{Procedure for interaction with subcontractor}
The subcontractor, known as \emph{Company C}, can be contacted via email at the following address in order to arrange meetings, exchange documents, etc..\\\\

Company C: riccardodl92@gmail.com\\\\

\noindent The subcontractor can contact Problematic via email at the follow address in order to arrange meetings, exchange documents, etc..\\\\

Problematic: kristoffersg@hotmail.com\\\\

\noindent The interactions between the two parties, Company C and Problematic should follow the structure given below:
\begin{enumerate}
    \item Work will be initiated by Problematic delivering a Statement of Work along with a series og requirements for the subsystem, called \emph{AreaAware Dismounted}, of AreaAware that Company C are to deliver.
    \item Based on the Statement of Work and requirements for AreaAware Dismounted, Company C is expected to deliver a proposal of System Requirement Specification(SRS).
    \item Problematic will go over the proposed SRS and call for a meeting to evaluate the SRS.
    \begin{enumerate}
        \item This can lead to a repeat of step 2. and 3. until an agreement have been reach.
    \end{enumerate}
    \item When an agreement upon the SRS have been reached Company C can begin designing the AreaAware Dismounted system and is expected to deliver a Detailed Design Document(DDD) to Problematic.
    \item Problematic will go over the proposed DDD and call for a meeting to evaluate the DDD.
    \begin{enumerate}
        \item This can lead to a repeat of step 4. and 5. until an agreement have been reach.
    \end{enumerate}
\end{enumerate}
The structure given above are some strict points that the interaction must follow.
However correspondence regarding the AreaAware Dismounted system, which are not specified within this structure are allowed, but should also go through the prober emails.\\\\
All formal meetings held between the parties must be documented by a Minutes of Meeting distributed to both parties.


%% Work Breakdown Structure
\section{Work Breakdown Structure}:
\subsection{Work Breakdown Structure}:
\begin{table}[]
\centering
\caption{Work Break Down Structure for System Product - AreaAware}
\label{tab:wbs_sysproduct}
\begin{tabular}{lllll}\hline
Level 1 & Level 2 & Level 3 & Level 4 & Level 5 \\\hline
1. AreaAware   &             &           &             &         \\
            & 1.1 AreaAware  &           &             &         \\
            &                &  SA-AA-HQ &             &         \\
            &                &           & SA-AA-HQ-WP &         \\
            &                &           & SA-AA-HQ-CM &         \\
            &                &           &             & TBD     \\
            &                &  SA-AA-SE &             &         \\
            &                &           & SA-AA-SE-UI &         \\
            &                &           & SA-AA-SE-DB &         \\
            &                &           & SA-AA-SE-LS &         \\
            &                &           &             & TBD     \\
            &                &  SA-AA-DI &             &         \\
            &                &           &             &         \\\hline
\end{tabular}
\end{table}

%% SECTION Requirementes Analysis
\section{Requirements Analysis}
The requirements are specified in document AreaAware-SRS-20160214-01-02 \& AreaAware-DISSRS-20160214-01-02.


%% SECTION Functional Analsyis
\section{Functional Analysis}
The functional analysis are specified in section 3.2 of document  AreaAware-PDD-20160214-01-01.


%% SECTION: System Analysis and Control
%\section {System Analysis and Control}
