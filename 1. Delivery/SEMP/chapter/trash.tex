% %!TEX root ../semp.tex
% %
% \thispagestyle{fancy}
% \chapter{Integration of the System Engineering Effort}
% \label{chp:se_process}

% %% SECTION Systems engineering process
% \section{Systems engineering process}

% \subsection{Deliverables}
% \label{plan:deliverables}
% The milestone plan here will denote the deadline the delivery of system engineer documents for the AreaAware project, to its proper stakeholders.
% A review will ensure all partners are satisfied with content of these documents.
% In case of change of documents content, a new delivery and review should should take place.
% \myTable{Plan for delivery of system engineer documents. The colored cells denote {\color{deadlinecolor} Deadline} and \color{reviewcolor} Review}{tab:milestones}{table/milestone}
% % \section{Statements of Work (SOW)}
% % The main purpose of the SOW is to define liabilities, responsibilities and work agreements between client and service providers.

% % \subsection{Introduction}
% % \label{sec:sow_intro}
% % In a military battlefield the \emph{SitaWare Suite} provides the capability to establish situational awareness, what in other settings may be referred to as a common operations picture (COP).
% % The project AreaAware wishes to extend these possibilities offered by \emph{SitaWare Suite} to cover civilian scenarios, where actors for police, armed forces, hospitals and emergency management work together to control a crisis situation.
% % A successful result of this could be a scenario where the commanders from a mobile headquarter can evaluate incoming information, act upon it and possibly dispatch orders or information to relevant actors to respond to by.

% % \subsection{Objectives}
% % The objective of Project AreaAware is to deliver a solution capable of fulfilling the successful result described above in \ref{sec:sow_intro}.

% % \subsection{Scope of Work}
% % The scope of AreaAware
% % \todojens{inline}{Mere om hvad}


% %% SECTION Requirmentes Analysis
% \section{Requirments Analysis}




% %% SECTION: System Analysis and Control
% \section System Analysis and Control

% \section{Interface Plans}
% \myTable{Interface Elements}{tab:interfaceElements}{table/interfacePlans.tex}



% \section{Responsibilities}

% \section{Procedures}

% \section{Authorities}

% \section{Work Breakdown Structure (WBS)}





% \section{Schedules}

% \section{Program Events}

% \section{Program Technical}

% \section{Test Readiness Reviews}

% \section{Technical and Schedule Performance Metrics}

% \section{Engineering Program Integration}

% %!TEX root ../semp.tex
% %
% \thispagestyle{fancy}
% \chapter{System Engineering Process}
% \label{chp:se_process}


% \section{Mission}

% \section{System Overview Graphic}

% % Requirement and Functional Analysis
% % Trade Studies (Analysis of Alternative
% % Technical interface and analysis/planning
% % Specification Tree/Specifications
% % Modeling and Simulation
% % Test Planning
% % Logistics support analysis
% % System engineering tools
% \section{Organization Structure}
% The development of AreaAware will be managed by Problematic.
% Hence, it is Problematic that will denote the organization of system engineering through out the project.
% The system engineering documents will describe what, when and why different artifacts are expected of the proper stakeholders.
% All these documents are provided by Problematic through out the project, expected by the date denoted in the Milestone plan, section \ref{plan:milestone}.\\


% \noindent AreaWare product will be designed by Problematic in cooperation with the projects stakeholders.
% The full product will be developed by development team of Problematic and by one subcontractor.
% To ensure that the project are following the norms and requirments listed in system engineering documents, both team will have a system engineer associated.

% %The system engineer role is to ensure the project are following all the norms and requirements listed in the SEMP and \todo{Kravspec}, aswell to support all members of the project regarding design and process.
% %The system engineer should also continue write and update all relevant system engineering document for the project.

% %The development team role is to implement AreaAware as it is described by the system engineer(s).
% %The team consists of various fields, including software architects, software engineers and test engineers.\\

% % Subcontractor
% %The subcontractor will have one system engineer from Problematic to ensure projects are following all norms and requirements listen the SEMP and \todo{Kravspec}, aswell supporting in case of questions regarding the design and process of the project.


% % \section{Statements of Work (SOW)}
% % The main purpose of the SOW is to define liabilities, responsibilities and work agreements between client and service providers.

% % \subsection{Introduction}
% % \label{sec:sow_intro}
% % In a military battlefield the \emph{SitaWare Suite} provides the capability to establish situational awareness, what in other settings may be referred to as a common operations picture (COP).
% % The project AreaAware wishes to extend these possibilities offered by \emph{SitaWare Suite} to cover civilian scenarios, where actors for police, armed forces, hospitals and emergency management work together to control a crisis situation.
% % A successful result of this could be a scenario where the commanders from a mobile headquarter can evaluate incoming information, act upon it and possibly dispatch orders or information to relevant actors to respond to by.

% % \subsection{Objectives}
% % The objective of Project AreaAware is to deliver a solution capable of fulfilling the successful result described above in \ref{sec:sow_intro}.

% % \subsection{Scope of Work}
% % The scope of AreaAware
% % \todojens{inline}{Mere om hvad}