%!TEX root ../semp.tex
%
\thispagestyle{fancy}
\chapter{Introduction}
\label{chp:intro}


\section{Scope}
This document is applicable to the AreaAware system, all subsystems and component engineering efforts accomplished throughout the case.

\section{Purpose}
The development of a large system is often complex and involves a lot of participants.
In order for the project to be carried out smoothly it is essential that all of the participants in the system development know their responsibilities.
Not only for their own part of the system, but also their responsibilities and interfacing to one another.

The System Engineering Management Plan(SEMP) plans the implementation of all systems engineering tasks and defines the interfacing of responsibilities and authority of all participants(subsystem managers, component design engineers, test engineers, system analyst, specialty engineers, subcontractors etc.) within the project.

The SEMP serves as a reference for the procedures that are to be followed in carrying out the numerous systems engineering tasks in the course of system development and serve as an instrument for comparing the planned tasks with the tasks accomplished.

The document is intended to be a living document, starting as an outline and constantly being elaborated as the system development goes on.

\section{Referenced Documents}



