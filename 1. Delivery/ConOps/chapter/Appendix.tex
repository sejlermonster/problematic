\appendix
\label{chp_appendix}
\chapter{Glossary and Terms}

\textbf{COP:} The AreaAware system is providing a Common Operations Picture(COP) to give the involved commanders situational awareness.\\

\noindent \textbf{Mobile HQ COP:} This is the main part of the AreaAware system. The commanders in the EHS can connect their equipment the Mobile HQ COP to get situational awareness.\\

\noindent \textbf{Dismounted COP:} Consist of a condensed COP on a handheld device for commanders in the field.\\

\noindent \textbf{EHS:} The Emergency Headquarter Station is the center of command in a given disaster situation.\\

\noindent \textbf{ECs:} The Emergency Commanders is the personal in the EHS who is leading the disaster situation.\\

\noindent \textbf{FCs:} The Field Commanders is scattered around the disaster area and using the Dismounted COP for communication and coordination.\\

\noindent \textbf{ERs:} The Emergency Responders is a collection of disaster management departments e.g. police, armed forces, hospitals, fire departments and ect.\\

\noindent \textbf{Static information:} Is data about the demographics together with static information like fresh water supplies, power lines, infrastructure placements and ect. of the area affected a disaster.\\

\noindent \textbf{Dynamic information:} Is data like actor positions, actor observations, weather conditions, damaged infrastructure sections and ect. of the area affected a disaster.\\

\noindent \textbf{SINE:} Separated communication network in Denmark for communication between the different emergency departments. It is designed function for everyday use and under large-scale civil emergencies.\\