\chapter{Introduction}
\label{chp_intro}

\section{Purpose}
The purpose of the Concept of Operations (ConOps) is to describe the capabilities, missions and support scenarios of the suggested system. It fills gaps between the Mission Needs Statement (MNS) and the System Requirements Specification (SRS).
Furthermore, the CONOPS informs the mission managers, project management staff, developers, operational support staff, users and other stakeholders of the intended uses or capabilities of the system. It enables an early assessment of the fit of a solution to an operational environment and its’ expected performance in achieving missions and fulfilling needs.

\section{Executive summary}
The AreaAware system is used is case of large-scale civil emergencies such as terrorist attacks, natural disasters or disease/toxic outbreaks, by providing a Common Operations Picture (COP). The HQ commanders can use the COP to get an overview of the complex situation and issue orders to deal with the situation in the best way. Field commanders can also use the systems condensed COP to make decisions and issue orders in the field. Both the HQ commanders and Field commanders use the communication network SINE to communicate with the Emergency Responders such as police, armed forces, hospitals and fire departments. The system have to be ready for deployment 24/7 and have be reliable in environments effected by major disasters.

\section{Revision summary}

\begin{tabular}{b{1cm} b{2cm} b{2cm} b{7cm}}
    \textbf{Ver.} & \textbf{Init.} & \textbf{Date} & \textbf{Note} \\
    \hline
    0.01 & SBG		& 2016-02-01 & Startup work \\
    0.5  & JBM/SBG	& 2016-02-04 & Initial draft made in google docs \\
    0.6  & JBM/SBG	& 2016-02-05 & Acquired information on systematics SitaWare. Revised written chapters \\
    0.9  & JBM/SBG	& 2016-02-08 & Google docs version finished, except summary \\
    1.0  & JBM/SBG	& 2016-02-09 & ConOps is written in LaTex \\
\end{tabular}