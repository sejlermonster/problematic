\chapter{Operations and Support Description}
\label{chp_operations}
This chapter describes the missions, users and general operation and support description of the system.

\section{Missions (Primary/Secondary)}
Primary missions:
\begin{enumerate}
	\item To gather information and events from multiple Emergency Responders to establish situational awareness.
	\item Display large amounts of static and dynamic data to give a COP for the Emergency Commanders and Field Commanders to make the right decisions.
	\item Dispatch orders from the Emergency Commanders or Field Commanders to the different Emergency Responders to deal with the given situation. Send messages between Emergency Commanders, Field Commanders and Emergency Responders.
\end{enumerate}

\noindent Secondary users:
\begin{enumerate}
	\setcounter{enumi}{3}
	\item Deploy Dismounted COPs for Field Commanders to make individual decisions.
\end{enumerate}

\section{Users and Other Stakeholders}
\noindent Primary users:
\begin{itemize}
	\item Emergency Commanders (ECs): \\
	These Commanders are located in the mobile Headquarters, and are able to utilize the full capabilities of the COP. They can receive events registered using employed platforms from various actors and send information to these platforms.
	\item Field Commanders (FCs):\\
	Emergency Commanders who are located in the field. They will have access to a condensed version of the COP on a hand-held device, called the AreaAware Dismounted.
\end{itemize}
	
\noindent Secondary users:
\begin{itemize}
	\item Emergency Responders (ERs):\\
	Are able to register events using the currently employed platforms. They are also able to receive information on the same platforms. The Emergency Responders include the police, armed forces, hospitals, emergency management and etc.
\end{itemize}

\noindent Stakeholders:
\begin{itemize}
	\item The State/society: \\
	Is interested in the system as the State will likely be the ones utilizing the system, and its capabilities to handle an emergency situation. The entire system is being implemented to provide a better service in emergency situations for society.
	\item Subcontractor:\\
	The subcontractor is tasked with developing the AreaAware Dismounted for the Field Commanders.
\end{itemize}


\section{Policies, Assumptions and Constraints | Policy, Assumption, Constraint}
\textbf{Policies} \\
A common policy for communication will have to be provided for secondary actors.. 
Several policies, laws and restrictions that are in place under everyday situations for each of the secondary actors are suspended under the emergency situations where the system is to be used. Therefore, these are not taking into account in this document.\\

\noindent \textbf{Assumptions}
\begin{enumerate}
	\itemsep0em
	\item The subcontractor will deliver a satisfactory product, satisfying the specified requirements for the AreaAware Dismounted.
	\item It is possible for the Emergency Responders to implement features making it possible to register events to and receive orders from ECs.
	\item The AreaAware HQ, hand-held devices of the FCs and the Emergency Responders’ separate systems are compatible with the SINE network.
	\item Connection can be established to the SINE network for all relevant devices, allowing communication.
\end{enumerate}

\noindent \textbf{Constraints}
\begin{enumerate}
	\item The system is developed for Denmark, where the SINE network is deployed.
\end{enumerate}

\section{Operation Description | Operating Concept (OpCon), Employment Modes}

This section describe the environment in which the system is operating, and interfaces to the surrounding actors and subsystems.

\subsection{Operational Concept}
The figure below show the overall principle of the AreaAware system, and how it is connected the various users and subsystems.

\begin{figure}[ht]
	\centering
	\includegraphics[width=1\linewidth]{figure/OpCon.png}
	\caption{Operating concept, showing the major players, systems and subsystems and their interrelationships}
	\label{fig:OpCon}
\end{figure}

\FloatBarrier

\noindent The AreaAware system contain two parts, the AreaAware HQ and the AreaAware Dismounted. The AreaAware HQ provide situational awareness for the EC's inside the EHS, and the AreaAware Dismounted provide situational awareness for the FC's at various positions inside disaster zone. AreaAware use the SINE network for communication with the ER's. This way the EC's and FC's are able to get information and issue orders to any part of the personnel involved in resolving a disaster situation.

\subsection{Employment Modes}
\textbf{Idle:} The system is not in active use, but is ready for fast deployment is case of a large-scale civil emergency.\\

\noindent \textbf{Deployment:} The AreaAware HQ is deployed with the EHS in the area affected by a disaster, and communication is established with all involved ER's and FC's using SINE.\\
 
\noindent \textbf{Live disaster management:} Gather information from the ER's, present these data for the EC's and FC's, and send commands to the ER's for solving the disaster.\\

\noindent \textbf{Emergency power mode:} A mobile power source is used, when the general power grid is offline.

\subsection{Scheduling and Operations Planning}
The AreaAware HQ and AreaAware Dismounted will be a part of the Danish emergency Management Agency’s mobile equipment for disaster handling, and will have to be ready for deployment 24/7.

\subsection{Operating Environment}
\noindent \textbf{Geographic Area:}\\
The AreaAware system can be deployed in any geographic area affected by a major disaster, but for simplicity it is constrained to be in Denmark. \\

\noindent \textbf{Environmental Conditions:}\\
The AreaAware HQ is located with the ECs inside the EHS. The vehicle/station will shield the system against bad weather conditions and ect.

The AreaAware Dismounted can experience a wide range of environmental scenarios. This include temperature ranges, humidity levels, weather conditions and ect. The FC's is located close to the action inside the disaster zone. It is therefore important that the FC can get a overview and make quick decisions in stressful situations.

\subsection{Threats and Hazards}
As AreaAware is deployed in a disaster situation there are potential hazards relating to the environment. It is likely that the system will experience extreme weather situations like hurricanes and floods.

Data security is important for the communication between the actors in the system. It would e.g be a problem if a group of terrorist could track the police’ movements by hacking the AreaAware communication network.

\subsection{Interoperability with Other Elements}
The AreaAware HQ have a general interface for a wide range of equipment, for supplying situational awareness for the ECs. The AreaAware HQ also have an interface to the SINE network and an internet connection. The AreaAware Dismounted will also need an interface to the SINE network for communication.

\newpage
\section{Product Support Description}
This section will describe the support plan of the system, including the extent and duration of support and the existence of live-support. 

\subsection{Software Warranty}
A Software Warranty Service is part of the Statement of work for the system \cite{Casebook}. The Software Warranty Service is cited below: \\

\hfill\begin{minipage}{\dimexpr\textwidth-1cm}
\textit{"Software Warranty Service consists of:
Defect Reporting. For Critical Defects, the Customer will have 24x7 access to the Service Centre by e-mail or phone to request defect repair, as described below. “Critical Defect means that the application is down or is at high risk, business functions cannot be conducted, or the Customer is experiencing continual failures or data corruption as a result of the defect. To report non-critical defects, the Customer will have e-mail or phone access to the Service Centre during the Principal Period of Maintenance (”PPM”), which is 8:00 a.m. to 5:00 p.m., local time, Monday through Friday, excluding local holidays.} \\

\textit{Defect Repair. Defect repair includes verification of the existence of a defect, determination of the severity or impact of the defect, and determination of the conditions under which the defect may recur. The Company will, at its option:}
\begin{itemize}
	\itemsep0em
	\item \textit{For a Critical Defect, commence action within a 2-shift hour response window using  commercially reasonable efforts to provide an immediate fix or temporary solution of, or workaround to, the defect.}
	\item \textit{For a non-critical defect, commence action within an 8-shift hour response window to provide either the action described for a Critical Defect or a statement that the defect will be corrected in a software product revision or a future software release.}
	\item \textit{Provide a statement that the Software operates as described in then-current user documentation or that the defect arises when such Software is used other than in a manner for which it was designed. For Software added to an installed System, warranty service must be upgraded to the same software support plan, if any, as that of the Software already installed on that System. Customer will pay the difference between standard warranty and upgraded warranty service."}
\end{itemize}
\end{minipage} \\

\noindent Furthermore, an extended 24/7 support service is offered to the customer, where the contractor is continually offering training, equipment and information support.
The Six Facets of Readiness \cite{Six_facets} https://www.uscg.mil/acquisition/newsroom/pdf/msam.pdf will be used to describe the support of the system:

\subsection{People}
As the system is not directly involved with any physically demanding task, there is no need to consider the medical care of the users of this system.

\subsection{Training}
The users of this system will have a user manual to refer to, describing all the features of the system in a user-friendly manner. If need be, the user can call our 24/7 support with questions.

\subsection{Equipment}
The configuration control of the system is maintained by the actors, with cooperation of the CSP. If configuration control of the system is not maintained, a user manual, as well as 24/7 support according to the Software Warranty. 

\subsection{Support}
In accordance with the Software Warranty, access is available 24/7 to the service center, where the customer can request defect repair. There are two types of defect requests:
\begin{itemize}
	\itemsep0em
	\item Critical Defect, for which an action must to commenced within a 2-shift hour response window, using reasonable efforts. This action must provide an immediate fix, temporary solution or workaround to the defect. This is available for 24/7 support.
	\item Non-critical defect, where an action is required to commence within an 8-shift hour response window. This action must provide the service as for a Critical Defect or a statement has to be released describing that it will be fixed in a future release. \\
	The non-critical defect can be reported during the Principal Period of Maintenance, as described in the Software Warranty.
\end{itemize}

\subsection{Infrastructure}
The current infrastructure will support the system. In emergency situations, where the system has to be deployed, precautions will be made to ensure the required infrastructure.
In case of an earthquake, or other disaster, making conventional infrastructure virtually inaccessible, the AreaAware HQ can be transported by helicopter etc. to the area where it is needed.

\subsection{Information}
Information will be maintained by the development team, who will keep supporting and maintaining the system for an agreed-upon amount of time.


\section{Potential Impacts}
This system have the potential to significantly improve the emergency response service provided by various actors in a large-scale civil emergency situation. This is done by providing intelligence for assigned commanders in the form of a COP and giving them the possibility to receive events from and dispatch orders to ERs. ERs will have to implement a feature in the respective currently employed systems in order to enable communication with the suggested system. 
The suggested system will be compatible with the SINE network in Denmark, securing a stable connection between devices even in severe emergency situations, where other networks fail.


\newpage
\section{Scenarios, Capabilities Needed}
\label{sec_scenarios}
This section will describe the scenarios used	in the life of this system. These scenarios will be based on scenarios describing functional capabilities of the system. The functional capabilities are described further in chapter \ref{chp_capabilities}.

\subsection{Mission operations scenarios}
\noindent \textbf{Scenario 1: Civil emergency situation has just started} \\
The system is on standby, waiting to be deployed. 
A large-scale civil emergency situation has occurred and the AreaAware HQ is to be deployed in the area of crisis. An EHS is set up, wherein the ECs are present. The AreaAware HQ unit is deployed in this station, providing the ECs with a COP. The ECs each connect a computer to the AreaAware HQ, after which a COP displaying relevant information is available.\\
Figure \ref{fig:S1_flow} show the functional flow of this scenario. \\

\noindent Prerequisites:
\begin{itemize}
	\item The AreaAware HQ is in a safe facility, where relevant personnel can access it 24x7 if needed.
\end{itemize}

\noindent Functional Capabilities needed:
\begin{itemize}
	\itemsep0em
	\item AreaAware HQ.
	\item Provide a COP.
\end{itemize}

\myFigure{functional_flow_S1}{Functional flow block diagram of scenario 1}{fig:S1_flow}{1}

\newpage
\noindent \textbf{Scenario 2: AreaAware HQ is in operation} \\
The ECs are set up in the area and have access to a COP from the AreaAware HQ.
From the COP, static and dynamic information in the area is displayed and the ECs are provided a real-time picture of the area of crisis, including events send in by various ERs and positions of FCs.
The commanders have the ability to focus on various sources and types of information in the COP and review the history of one or more events. 

The commanders additionally have the capability to dispatch orders and messages to ERs and other commanders directly from the COP, through the SINE network. The commanders have the ability to control the distribution of messages based on geography, role, group and identity. Figure \ref{fig:S2_flow} show the functional flow of this scenario. \\

\noindent Prerequisites:
\begin{itemize}
	\itemsep0em
	\item The AreaAware HQ unit is in deployed in an EHS, located in the vicinity of the emergency situation.
	\item The SINE network is functional and accessible. Connection is established through this network.
\end{itemize}

\noindent Functional Capabilities needed:
\begin{itemize}
	\itemsep0em
	\item Data collection.
	\item Information focusing.
	\item SINE network capability.
	\item Message distribution control.
\end{itemize}

\noindent Functional Capabilities Delivered by Alternatives:
\begin{itemize}
	\item Emergency Responders have to integrate a function to receive information/orders and send events to the AreaAware HQ via the SINE network.
\end{itemize} 

\myFigure{functional_flow_S2}{Functional flow block diagram of scenario 2}{fig:S2_flow}{1}

\newpage
\noindent \textbf{Scenario 3: Operation of handheld devices by Field Commanders} \\
an AreaAware HQ is in operation. Communication through the SINE network has been established and the ECs are provided with a COP. 

The FCs each have an AreaAware Dismounted unit, which provides them with a condensed version of the AreaAware HQ COP. This provides them with static and dynamic information along with the position of other commanders in the area of emergency. From the COP they are able to communicate with EC's and receive events from ERs.\\
Figure \ref{fig:S3_flow} show the functional flow of this scenario. \\

\noindent Prerequisites:
\begin{itemize}
	\itemsep0em
	\item The Field Commanders have a hand-held device with the necessary software installed.
	\item Connection with other commanders is  established through the SINE network.
\end{itemize}

\noindent Functional Capabilities needed:
\begin{itemize}
	\item AreaAware Dismounted.
\end{itemize}

\myFigure{functional_flow_S3}{Functional flow block diagram of scenario 3}{fig:S3_flow}{1}

\newpage
\subsection{Support operations scenarios}
\noindent \textbf{Scenario 1: Repair of critical defect}
As described in the Software Warranty Service.
The operating personnel calls the 24/7 defect support. The support staff decided on a course of action and initializes it as soon as possible. An action is commenced within a 2-shift hour response window using commercially reasonable efforts. This action must have an immediate fix, temporary solution to, or workaround of, the defect.

\noindent Functional Capabilities Needed:
\begin{itemize}
	\item 24/7 support service \& call-center.
\end{itemize}

\vspace{20pt}
\noindent \textbf{Scenario 2: Repair of non-critical defect}
As described in the Software Warranty Service.
The operating personnel calls the defect support within the Principal Period of Maintenance. The support staff commences an action within an 8-shift hour response window. This action must provide the same action as for a critical defect or a statement that it will be fixed in a future release. \\

\vspace{20pt}
\noindent \textbf{Scenario 3: Operational/configurational support}
The operating personnel calls the 24/7 extended support service and describes the problem. The supporting staff will either provide support over the communications device, guiding the operating staff through a solution, or remotely control the operating personnel’s computer to provide a solution. If an immediate solution is not possible to achieve from afar, a member of the support staff will be dispatched where support is needed. 

\noindent Functional Capabilities Needed:
\begin{itemize}
	\item Extended 24/7 support service
\end{itemize}













