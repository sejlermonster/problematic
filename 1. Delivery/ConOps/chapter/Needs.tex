\chapter{Capability Needs}
\label{chp_need}

This chapter will describe which capabilities are needed for the system in question to implemented, along with how the current situation is.

\section{Business Need(s)}

The purpose of AreaAware is to establish situational awareness, by creating a Common Operations Picture (COP), during large-scale civil emergencies such as terrorist attacks, natural disasters or disease/toxic outbreaks. These situations can impact multiple parts of the society and can cause breakdown of telecommunication networks, blocked infrastructure, power blackouts, panic in the affected population and ect. The system have to be deployed fast in case of a disaster, to quickly establish this situational awareness that is necessary to deal with the situation. The COP will consist of both static and dynamic data, displayed in a map of the local area, which gives the commanders a quick overview of the Emergency situation. \\

\noindent The system have certain needs to deal with these situations.

\begin{itemize}
  \item A mobile headquarter to provide a COP, which is needed to show and evaluate incoming information, act upon it and possibly dispatch orders or information to relevant actors.
  \item A Dismounted COP to provide situational awareness in the field, which enables the commanders to act independently of the mobile HQ.
  \item A communication network with emergency services such as police, armed forces, hospitals and fire departments is needed to register events and send orders.
\end{itemize}

\section{(Business Need) Capability Gap}

The Danish emergency Management Agency have a mobile command vehicle, which is able to do nationwide communication with the SINE network to solve a disaster situation, but they lack the means of getting a good situational awareness. All the static information about the disaster environment and the dynamic data from the Emergency Responders in the field, have to be displayed to give the best possible situational awareness, for the commanders to make the best dissections. This is both needed for commanders in the Emergency Headquarter Station(EHS) in the form of COP, and for the Field Commanders in the form of condensed COP on a handheld device.

\section{Current situation}

In Denmark a large-scale civil emergency will be handled by the Danish emergency Management Agency. This agency will handle the coordination of all the Emergency Responders such as police, armed forces, hospitals and fire departments, to ensure that the available resources is used in the best way to handle the current situation. The agency have a mobile command and communication vehicle called a LKM. This vehicle is manned by commanders and specialists, and can be deployed anywhere in Denmark with hours notice.

A system called SINE is used for joint communication between all the emergency departments. SINE is a radio based system with 500 antennas to cover all Denmark. The SINE network is handled by a central station called the SWITCH, which is handling all the communication. The SINE network is designed to handle both everyday communication and big emergency situations, and all communication is encrypted. SINE has emergency power supplies, in case of power failures. All in all SINE is widely used with around 15,000 active radios in Denmark, and is very robust in emergency situations. SINE is based on the european communication standard called Tetra, which is used or under implementation in most of Europa.