\label{chp_revisionHistory}
\chapter{Glosary and terms}

This chapter will elborate on the meaning of specifc glosaries and terms used throughout this document. The chapter is fundamental for for understanding the following chapters.

\begin{longtable}{| p{3.5cm} |  p{10cm} | }
	\hline
	\textbf{Glosary og term} &  \textbf{Explanation } \\
	\hline
	COP & COP is used as an acronym for \emph{common operational picture}. A COP is a display of relevant shared information to the commanders of an operation. This facilities collaborative planning and assists in achieving situational awareness \\
	\hline
	SINE & SINE is and abbreviation of \emph{Sikkerhedsnettet}.This is the Danish standard for communication during civil crisis. SINE is based on the international standard for communication during a civil crisis called TETRA\\
	\hline
	Emergency management actors & Relevant emergency personal who is included in the work during a civil crisis. This could include hospital personal, police, armed forces ect.  \\
	\hline
	TA-X & Task Description no. X \\
	\hline
	NF-X & Non-functional requirement no. X \\
	\hline
	TC-X & Test case no. X \\
	\hline
\end{longtable}