\chapter{System-wide Design Document}
\label{chp:system-wide}

This chapter will describe the system-wide design decisions for the AreaAware system. The system will be regarded as a connection of black boxes, where non of the internal components are described.

\section{Input and output decisions}
Figure \ref{fig:in_out} describes the inputs and outputs of the AreaAware system as a whole. All of the system's actors have the ability to provide an input and to receive an output.\\

\myFigure{Inputs_outputs}{Figure describing the inputs and outputs of the system and the actors involved for each input or output}{fig:in_out}{1}

\subsection{Inputs}
The inputs of the system are as follows:

\begin{itemize}
	\item \textbf{Events:} An Event is information into the system, consisting of time, position, occurrence and/or messages. Each Event can be simple, such as positional updates of FCs, or complex such as the state and position of a local fire and the surrounding area. All Events provide Emergency and Field commanders with some kind of situational awareness, through an update on the COP, description of an area, etc. \\
	An Event Log keeps track of all events above a certain Level of Importance, which assigned to each event.
	\item \textbf{Command Decisions:} A Command Decision is a decision done by an EC or a FC, which results in a command.
\end{itemize}

\subsection{Outputs}
The outputs of the system are as follows:

\begin{itemize}
	\item \textbf{Situational Awareness:} Knowledge about the current emergency situation. Can consist of positional, time or general information about any situation or occurrence relevant to the emergency situation.
	\item \textbf{Info:} Information from a FC or EC on any current relevant situation, such as information about a fire to a group of firemen.
	\item \textbf{Commands:} The actual command, send by an EC or FC to a ER or FC. A Command could be specifying a specific unit where to move or how to handle a situation.
\end{itemize}

\newpage
\section{Data appearance}
Appearance of data for the user will happen on a User Interface on a screen. For the FC, the screen is on the AreaAware Dismounted, which provides a condensed COP in a native app. For the EC, the screen is a monitor, which is connected to a computer, running the AreaAware Web Interface. Figure \ref{fig:UI} shows an example of how the User Interface for the AreaAware Web Interface could be designed. 

\myFigure{UI_sketch}{Example of user interface for the AreaAware Web Interface. In the map on the right side, several icons describe the positions of several events and actors. The red fire and hazard icons describe the locations of a local fire and biohazard, respectively. The black firetrucks, ambulances and police car represents the locations of local emergency response vehicles, from which a dotted line represents where they a dispatched. The black plus-sign and HQ-sign represent the respective location of a hospital (or emergency clinic) and Emergency Headquarter Station.
\newline On the left side, an Event Log is visible, describing important ongoing events. The ECs and FCs can set the minimum Level of Importance required to be displayed on the Event Log. Each event can be accessed to view the history of that Event.
\newline Three buttons can be used by the commanders to create an Event, send a Command or send a message. These functionalities can also be accessed intuitively by clicking on the COP.}{fig:UI}{1}

\section{Safety, security and privacy}
The security of the data connection is very important for the system to work. Therefore, the system 






\section{Design and construction choices}
