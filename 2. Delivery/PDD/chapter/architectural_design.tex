%!TEX root = ../PDD.tex

\thispagestyle{fancy}
\chapter{Architectural Design}
\label{chp:architectural}

This chapter will describe the system architecture design.
The system architecture design is described by its components, concept of execution and interfaces.


\section{System Components}
AreaAware consists the main components:
\begin{itemize}
\item AreaAware Dismounted
\item AreaAware HQ
\item AreaAware Server
\end{itemize}
Figure \ref{fig:bbd_overview} shows the composition of one AreaAware system.
To one AreaAware system, consists at least of one AreaAware Server and AreaAware HQ, but not necessary a AreaAware Dismounted.
However in AreaAware system there can exists multiple of the three main components.

\myFigure{arch/aa_bdd_overview.png}{AreaAware consists the main components: AreaAware Dismounted, AreaAware HQ and AreaAware Server. In a AreaAware system can various AreaAware Dismounted and AreaAware HQ exists, AreaAware Server}{fig:bbd_overview}{1}

\noindent Each of these main components consists of subcomponents, which gives functionality for solving the requirements described in  AreaAware-SRS-20160214-01-02 \&  AreaAware-DISSRS-20160214-01-02.


\subsection{AreaAware Dismounted}
\begin{tabular}{l|*{2}{l}}
    \textbf{Block name}     & AreaAware Dismounted \\
    \textbf{Block ID}       & SA-AA-DI \\
    \textbf{Purpose}        &\multicolumn{2}{l}{\makecell[l]{Device for the Field Commander to interact\\ with AreaAware}}  \\
    \hline
    \textbf{Related}    & \textbf{ID} & \textbf{Document ID} \\
    Requirements & All & AreaAware-DISSRS-20160214-01-02 \\
    Design decisions & N/A & AreaAware-PDD-20160214-01-01 \\
\end{tabular}\\\\

\noindent In a crisis situation, the Field Commander will interact with AreaAware through mobile device called AreaAware Dismounted.
This devices will provide a COP (condensed) and possibilities for the user to communicate with the Emergency Commanders, Emergency Responders and other Field Commanders via audio, video and text messages.
AreaAware work with or without AreaAware Dismounted present in the system.
There can various AreaAware Dismounted connected to AreaAware at the same time.

The main design of AreaAware Dismounted is subcontracted to a subcontractor of Problematic.
This design is based on the requirements specified by Problematic in document  AreaAware-DISSRS-20160214-01-02.
The design architecture of AreaAware Dismounted will therefore not be specified here any further.

\subsection{AreaAware HQ}
\begin{tabular}{l|*{2}{l}}
    \textbf{Block name}     & AreaAware HQ \\
    \textbf{Block ID}       & SA-AA-HQ  \\
    \textbf{Purpose}        &\multicolumn{2}{l}{\makecell[l]{Main interaction for the Emergency Commander\\ with AreaAware}}  \\
    \hline
    \textbf{Related}    & \textbf{ID} & \textbf{Document} \\
    Requirements & All & AreaAware-SRS-20160214-01-02 \\
    Design decisions & \makecell[l]{D-HQ1, D-HQ2, D-HQ3,\\ D-HQ4} & AreaAware-PDD-20160214-01-01 \\
\end{tabular}\\\\

\noindent In a crisis situation, the Emergency Commander will manage the situation right at a (mobile or not mobile) headquarter.
AreaAware HQ is the service for Emergency Commander to interact with AreaAware at the headquarter.


\myFigure{arch/aa_bdd_hq}{AreaAware HQ consists of two subcomponents. A web site presented and a communication modem.}{fig:bbd_hq}{1}


AreaAware HQ consists of two subcomponents: a Web Presenter and a Communication Modem.
There can up to 14 Web Presenters within AreaAware HQ.
Only a single Communication Modem is sufficient.


\begin{tabular}{l|*{2}{l}}
    \textbf{Block name}     & Web Presenter\\
    \textbf{Block ID}       & SA-AA-HQ-WP  \\
    \textbf{Purpose}        &\multicolumn{2}{l}{\makecell[l]{Essentially a web browser showing the \\ AreaAware Web Application.}}  \\
    \hline
    \textbf{Related}    & \textbf{ID} & \textbf{Document} \\
    Requirements & \makecell[l]{TA-01, TA-02, TA-03, TA-04,\\ TA-05, TA-06, TA-07, TA-08,\\ NF-05, NF-06, NF-07, NF-08,\\ NF-09, NF-10, NF-11, NF-23} & AreaAware-SRS-20160214-01-02  \\
    Design decisions & D-HQ1, D-HQ2, D-HQ4 & AreaAware-PDD-20160214-01-01 \\
\end{tabular}\\\\

\begin{tabular}{l|*{2}{l}}
    \textbf{Block name}     & Communication Modem\\
    \textbf{Block ID}       & SA-AA-HQ-CM  \\
    \textbf{Purpose}        &\multicolumn{2}{l}{\makecell[l]{Manages communication with SINE or the commercial internet.\\Convert data packets if necessary.}}  \\
    \hline
    \textbf{Related}    & \textbf{ID} & \textbf{Document} \\
    Requirements & \makecell[l]{NF-01, NF-02, NF-03, NF-04,\\ NF-12, NF-13, NF-14, NF-15,\\ NF-16, NF-17, NF-18, NF-19} & AreaAware-SRS-20160214-01-02 \\
    Design decisions & D-HQ3 & AreaAware-PDD-20160214-01-01 \\
\end{tabular}\\\\


\begin{tabular}{l|*{2}{l}}
    \textbf{Block name}     & SINE Communicator\\
    \textbf{Block ID}       & SA-AA-HQ-CM-SC  \\
    \textbf{Purpose}        &\multicolumn{2}{l}{\makecell[l]{Manages communication with SINE}}  \\
    \hline
    \textbf{Related}    & \textbf{ID} & \textbf{Document} \\
    Requirements & \makecell[l]{NF-01, NF-02, NF-14, NF-15,\\ NF-17} & AreaAware-SRS-20160214-01-02 \\
    Design decisions & N/A & AreaAware-PDD-20160214-01-01 \\
\end{tabular}\\\\

\begin{tabular}{l|*{2}{l}}
    \textbf{Block name}     & \multicolumn{2}{l}{Commercial Internet Communicator} \\
    \textbf{Block ID}       & SA-AA-HQ-CM-CI  \\
    \textbf{Purpose}        &\multicolumn{2}{l}{\makecell[l]{Manages communication with the commercial internet\\such as LTE/4G or such.\\}}  \\
    \hline
    \textbf{Related}    & \textbf{ID} & \textbf{Document} \\
    Requirements & NF-16, NF-17 & AreaAware-SRS-20160214-01-02  \\
    Design decisions & N/A & AreaAware-PDD-20160214-01-01 \\
\end{tabular}\\\\



\subsection{AreaAware Server}
\begin{tabular}{l|*{2}{l}}
    \textbf{Block name}     & AreaAware Server \\
    \textbf{Block ID}       & SA-AA-SE  & \\
    \textbf{Purpose}        &\multicolumn{2}{l}{\makecell[l]{Provide AreaAware HQ, AreaAware Dismounted\\and Emergency Responder with data.\\Keeper of all system logistics.}}  \\
    \hline
    \textbf{Related}    & \textbf{ID} & \textbf{Document} \\
    Requirements & \makecell[l]{TA-01, TA-02, TA-03, TA-04,\\ TA-05, TA-06, TA-07, TA-08,\\ NF-01, NF-02, NF-03, NF-12,\\ NF-13, NF-14, NF-15, NF-16,\\ NF-18, NF-19} & AreaAware-SRS-20160214-01-02  \\
    Design decisions & N/A & AreaAware-PDD-20160214-01-01 \\
\end{tabular}\\\\


The AreaAware handles all data and logistics across all devices in AreaAware - under and after a crisis.



\myFigure{arch/aa_bdd_server.png}{AreaAware Server consists of three subcomponents: a database, a web API and a logging system.}{fig:bbd_hq}{1}



\begin{tabular}{l|*{2}{l}}
    \textbf{Block name}     & Database\\
    \textbf{Block ID}       & SA-AA-SE-DB  & \\
    \textbf{Purpose}        &\multicolumn{2}{l}{\makecell[l]{Contain and manage all relevant of AreaAware.}}  \\
    \hline
    \textbf{Related}    & \textbf{ID} & \textbf{Document} \\
    Requirements & \makecell[l]{TA-01, TA-02, TA-03, TA-04,\\ TA-05, TA-06, TA-07, TA-08} & AreaAware-SRS-20160214-01-02  \\
    Design decisions & N/A & AreaAware-PDD-20160214-01-01 \\
\end{tabular}\\\\


\begin{tabular}{l|*{2}{l}}
    \textbf{Block name}     & Web API\\
    \textbf{Block ID}       & SA-AA-SE-UI  & \\
    \textbf{Purpose}        &\multicolumn{2}{l}{\makecell[l]{Web API used by AreaAware HQ's Web Presenter.}}  \\
    \hline
    \textbf{Related}    & \textbf{ID} & \textbf{Document} \\
    Requirements & \makecell[l]{TA-01, TA-02, TA-03, TA-04,\\ TA-05, TA-06, TA-07, TA-08} & AreaAware-SRS-20160214-01-02  \\
    Design decisions & N/A & AreaAware-PDD-20160214-01-01 \\
\end{tabular}\\\\


\begin{tabular}{l|*{2}{l}}
    \textbf{Block name}     & Logging System\\
    \textbf{Block ID}       & SA-AA-SE-LS  & \\
    \textbf{Purpose}        &\multicolumn{2}{l}{\makecell[l]{System for logging events happend AreaAware across all crisis situations.\\Can be exported for external use.}}  \\
    \hline
    \textbf{Related}    & \textbf{ID} & \textbf{Document} \\
    Requirements & \makecell[l]{N/A} & AreaAware-SRS-20160214-01-02  \\
    Design decisions & N/A & AreaAware-PDD-20160214-01-01 \\
\end{tabular}\\\\



\section{Concept of execution}
\label{sec:concept_execution}

The main components interact each other. This section will describe how these interactions and how the components execute.\\\\
\noindent Emergency Commanders interacts with AreaAware HQ.
This could be event registration or a command decision or getting a situation awareness.
Data the AreaAware HQ's Web Presenter show, are stored and managed by AreaAware Server.
The AreaAware HQ has two ways of communicating to AreaAware Server, therefore given following two scenarios:
\begin{itemize}
\item AreaAware HQ's Communication Modem has valid connection via commercial internet.
\item AreaAware HQ's Communication Modem has not valid connection via commercial internet.
\end{itemize}
In scenarios where AreaAware HQ has connection via commercial internet, data will be transmitted using that between AreaAware Server and AreaAware HQ.
If not, then will data transmitted through SINE.
Figure \ref{fig:modem} that AreaAware HQ has a modem communicating either via. SINE or commercial internet. The modem it controlling the connections between the two connections.

\myFigure{arch/modem.png}{The AreaAware HQ has a modem for controlling the connections to the commercial internet or SINE.}{fig:modem}{0.5}

\noindent Same goes with data between AreaAware Dismounted and AreaAware Server.
In scenarios where the AreaAware Dismounted has valid connection via commercial internet, data between AreaAware Server and AreaAware Dismounted will be transmitted using that.
However it is likely that the AreaAware Dismounted will operate in situations where no commercial internet available.
In such scenario the data will be transmitted via SINE.\\

\noindent All combinations of above scenario can seen in figure \ref{fig:ce}.
Note that a emergency commander always communicate with AreaAware via SINE.

    \begin{figure}[ht]
        \centering
        \subbottom[{Scenario 1: Both AreaAware HQ and AreaAware Dismounted has connection to AreaAware Server via commercial internet.}\label{fig:ce_a}]%
            {\includegraphics[width=0.47\textwidth]{arch/ce_4.png}}\quad\quad
        \subbottom[{Scenario 2: AreaAware HQ has has connection to AreaAware Server via commercial internet. AreaAware Dismounted has connection to AreaAware Server via SINE.}\label{fig:ce_b}]%
            {\includegraphics[width=0.47\textwidth]{arch/ce_2.png}}
        \subbottom[{Scenario 3: AreaAware HQ has connection to AreaAware Server via SINE. AreaAware Dismounted has connection to AreaAware Server via commercial internet.}\label{fig:ce_c}]%
            {\includegraphics[width=0.47\textwidth]{arch/ce_3.png}}\quad\quad
        \subbottom[{Scenario 4: Both AreaAware HQ and AreaAware Dismounted has connection to AreaAware Server via server.}\label{fig:ce_d}]%
            {\includegraphics[width=0.47\textwidth]{arch/ce_1.png}}
        \caption{eThe connection to AreaAware Server can be done through commericial internet or SINE. Emergency Responder contacts always via SINE.}
        \label{fig:ce}
    \end{figure}



The AreaAware Server has connection to the commercial internet and by that a connection to SINE.
The AreaAware Server will serve data to the communication the client (AreaAware HQ or AreaAware Dismounted) is using.
If the AreaAware Server has no connection either, AreaAware is unusable.
However the AreaAware Server is not a single device, but more of function.
This functionality can be split in setups with multiple devices, to ensure no downtime on the AreaAware.


\section{Interface design}
\label{sec:arch_interface}
The interface design of the system is described in a separate document \emph{Preliminary Interface Control Document} with the document ID \emph{AreaAware-PICD-20160216-01-01}.
