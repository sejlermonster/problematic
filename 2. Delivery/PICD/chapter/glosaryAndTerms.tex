\appendix
\label{chp_appendix}
\chapter{Glossary and Terms}

This chapter will elborate on the meaning of specifc glosaries and terms used throughout this document. The chapter is fundamental for for understanding the following chapters.
\newline

\textbf{COP:} The AreaAware system is providing a Common Operations Picture(COP) to give the involved commanders situational awareness.\\

\noindent \textbf{AreaAware HQ:} This is the headquarters of the AreaAware system. The commanders in the EHS can use the AreaAware HQ to get situational awareness and access to the SINE network.\\

\noindent \textbf{AreaAware Dismounted:} Consist of a condensed COP on a hand-held device for commanders in the field.\\

\noindent \textbf{EHS:} The Emergency Headquarter Station is the center of command in a given disaster situation.\\

\noindent \textbf{ECs:} The Emergency Commanders is the personal in the EHS who is leading the disaster situation.\\

\noindent \textbf{FCs:} The Field Commanders are scattered around the disaster area and using the Dismounted COP for communication and coordination.\\

\noindent \textbf{ERs:} Emergency Responders is a collection of disaster management departments e.g. police, armed forces, hospitals and fire departments.\\

\noindent \textbf{SINE:} Abbreviation for \textit{SIkkerhedsNEttet}. Separated communication network in Denmark for communication between the different emergency department. Uses the international TETRA standard.\\