\label{chp_introduction}
\chapter{Introduction}
This document describes the procedures related to testing and verification of requirements for the AreaAware HQ system. The related requirements can be found in SRS document for the AreaAware HQ system (document ID: \srshq).

\section{Identification}
This document is applies to AreaAware.
A system which goal is to provide a tool for a better and more effective handling of a civilian crisis.

The system will provide the commanders with an overview of the crisis in hand through a common operational picture(COP), where every task to be resolved can be identified and the emergency responders locations can be tracked.
This should make it easier to handle and evaluate a civilian crisis. 
The system will also provide a better communication between the multiple Commanders and Emergency Responders during a crisis.

AreaAware will communicate over the defined standard network used in these scenarios.
In Denmark this network is called SINE which follows the international standard TETRA used for communication during crisis situations world wide.

AreaAware can be adapted to any specific network instead of SINE easily only by changing the interface of AreaAware.


\section{System overview}
The system of interest will communicate with the already employed SINE network used by the emergency management agency.

The system will facilitate the commander in:
\begin{itemize}
	\item Give a demographic overview of the crisis area.
	\item Let the Commanders locate events, things of interest and ERs in the area.
	\item Let the Commanders send commands to other Commanders and ERs in the area.
\end{itemize}

The system will let commanders plug their own laptop(not provided) to the AreaAware HQ.
By plugging their laptop to the AreaAware HQ they will be allowed access to the common operation picture through a web application. 
The web application will be accessed through a browser installed on the customers own laptop.


\section{Document overview}
This document will provide a verification overview in Chapter  \ref{chp:testOverview}. In this chapter different types of tests will described and also the responsible of each test. In chapter \ref{chp:testProcedure} procedures related to anomaly and requirement changes is stated. Chapter \ref{chp:verificationActivities} describe different activities related to the review process of verification. Chapter \ref{chp:verificationReporting} will state the format of the formal report that is to be written after verification is done.


\section{Referenced documents}
In this table all the referenced documents will be listed.

\begin{tabular}{b{6cm} b{7cm}}
	\textbf{ID} & \textbf{Document name} \\
	\hline
	\rtm & Requirement Traceability Document \\
	\hline
	\srshq & System Requirement Specification - AreaAware Hq \\
	\hline
	\srsdis & System Requirement Specification - AreaAware Dismounted \\
	\hline
	\ddd & Detailed Design Description \\
		\hline
\end{tabular}