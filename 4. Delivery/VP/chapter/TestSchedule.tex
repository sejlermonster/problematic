\chapter{Test Schedule}
\label{chp:TestSchedule}

The table below describe a time plan for testing the individual components listed in \ddd.
When all individual components are successfully tested, the integration tests can begin by combining components for testing.
The integration test will increase in block size until the total system is tested.
The test specification can be found in \std.

\begin{table}[ht]
    \centering
    \begin{tabular}{|l|l|l|l|}
        \hline 
        \textbf{Component ID} & \textbf{Test type} & \textbf{Test timespan} & \textbf{Test deadline} \\
        \hline
        SA-AA-HQ-WP & Component test    & 2 weeks & 2019-06-01 \\
        \hline
        SA-AA-HQ-CM & Component test    & 2 weeks & 2019-06-01 \\
        \hline
        SA-AA-SE-DB & Component test    & 2 weeks & 2019-06-01 \\
        \hline
        SA-AA-SE-UI & Component test    & 2 weeks & 2019-06-01 \\
        \hline
        SA-AA-SE-LS & Component test    & 2 weeks & 2019-06-01 \\
        \hline
        SA-AA-DI    & Integration tests & 5 weeks & 2019-09-01 \\
        \hline
        SA-AA-HQ    & Integration tests & 5 weeks & 2019-09-01 \\
        \hline
        SA-AA-SE    & Integration tests & 5 weeks & 2019-09-01 \\
        \hline
        SA-AA       & Integration tests & 5 weeks & 2020-01-01 \\
        \hline
    \end{tabular}
    \caption{Schedule for test of the different components. The test schedule is based on the time plan in \semp table 5.3}
\end{table}