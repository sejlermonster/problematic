%!TEX root ../minutesOfMeeting.tex
%
\thispagestyle{fancy}
\chapter*{Minutes of meeting}


\section*{Call to order}
This meeting was called by Problematic in order to clarify some questions regarding the statement of work that Company E delivered to Problematic for subcontracting of F-16 combat aircraft System.
\textbf{Location:} Problematic at Finlandsgade 22 8200 Aarhus N, Denmark.\\ Building: Edison Room E235.\\
\textbf{Date:} Friday 26-feb-2016. \\
\textbf{Time:} 09:15-09:45.
\section*{Attendees}
\textbf{From Problematic:} Jeppe Blixenkrone-Møller \& Kim Lykke Johansen.\\
\textbf{Company E:} Jaamac Hassan Hire \& Troels Hessellund Knudsen.\\\\ 
\textbf{Meeting Moderator \& Time keeper:} Kim Lykke Johansen. \\
\textbf{Minutes Taker:} Kim Lykke Johansen.

\section*{POD}
\begin{itemize}
    \item Requirement 1.1/1.2: Problematic raises question to the direction of the dispense, since sideways dispense could lead to a dispense of payload into the aircraft.\\
    \begin{itemize}
        \item Company E states that the sideways dispense is expected to have a downwards angle to avoid payload to collide with the aircraft.
    \end{itemize}
    \item Requirement 1.3: Problematic would like some dimensions of the T-hook it self, and not just the spacing between them.
    \begin{itemize}
        \item According to Company E the T-hooks should be standard equipment for the F-16 aircraft.
    \end{itemize}
    \item Requirement 1.4: Problematic would like some inner dimensions as well.
    \begin{itemize}
        \item Company E will deliver these dimensions to Problematic. Within these dimensions there should be room for a temperature sensor.
    \end{itemize}
    \item Requirement 1.6: Needs to be more specific as to how much is dimming is acceptable. This applies t 04-SEN-ERAN
    \begin{itemize}
        \item Company E will deliver some specifications.
    \end{itemize}
    \item The POD should be given some weight requirements.
    \begin{itemize}
        \item Company E will deliver some specifications for this.
    \end{itemize}
    \item Problematic would like to know more about the physical interface for cables to the POD.
    \begin{itemize}
        \item The two MIL-STD-1553-B Data Bus are two separate cables of the same type.
        \item The interface need to be air tight.
    \end{itemize}
\end{itemize}


\section*{Climate Control Unit}
\begin{itemize}
    \item Problematic would like to know where the CCU should be placed, since it's not a part of the POD it self.
    \begin{itemize}
        \item The CCU is expected to be embedded in the aircraft, thus it is not a subject to the air pressure on the outside of the aircraft, but have to be able to withstand the same G-forces as the POD. 
    \end{itemize}
    \item Problematic and Company E agree on a solution for the CCU to use airflow to control the temperature within the POD. Cold air can be used from the environment, where as hot air be extracted from the heat of the engine.
    \item Requirement 2.1/2.2: Problematic would like to know if the strict interval 5-75 degree Celsius are satisfactory.
    \begin{itemize}
        \item Company E agrees to this interval.
   \end{itemize} 
   \item The interface to the PCU is not clear to Problematic, is both 115 V AC 400 Hz and 28 V DC accessible?
   \begin{itemize}
       \item Company E states that both are accessible.
    \end{itemize}
\end{itemize}


\section*{Other business}
\begin{itemize}
    \item Problematic will deliver the DDD, at Wednesday 2nd of March.
\end{itemize}

